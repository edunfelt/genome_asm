The results of running the program are summarized in Figure 1. As it turns out, there are many contigs with large degrees, and similarly, there are many components that are very large, but the number of such contigs and components is largely insignificant compared to the number of contigs with smaller degree and components of smaller size. Hence it is almost impossible to get a good look at these contigs in an informative way. However, as we can see in the figure, the absolute majority of contigs have a degree less than 10, and the size of most components is less than 10. There is around 2.5 million contigs with degree 1, and 500,000 components with only two contigs. 
\begin{figure}[ht]
        \centering
        \includegraphics[width=0.7\textwidth]{data.png}
        \caption{Overview of the contig degrees and component sizes.}
        \label{fig:results}
\end{figure}

The largest contig degree is 968, and the larges component has over 2.5 million contigs. We also see that the mean degree is 6, while the mean component size 8, which confirms that most values are small with a few larger ones that increase the value.
